\documentclass[a4paper,10pt,twocolumn]{article}

\usepackage{amsmath}
\usepackage{amsfonts}
\usepackage{amssymb}
\usepackage{jcematcom}
\usepackage[utf8]{inputenc}
\usepackage{listings}
\usepackage[pdftex]{hyperref}
\usepackage{caption}
\usepackage{subcaption}
\usepackage{float}

\captionsetup[table]{name=Tabla}

\raggedbottom

\hypersetup{colorlinks,%
	    citecolor=black,%
	    filecolor=black,%
	    linkcolor=black,%
	    urlcolor=blue}


\title{Empanada Acelerada.}

\author{\\
\name Alfonso Teja Rodríguez \email \href{mailto:alfonsotejeda150605@gmail.com}{alfonsotejeda150605@gmail.com}
	\\ \addr Grupo C212 \AND
\name Miguel Cazorla Zamora \email \href{mailto:miguelzamora210405@gmail.com}{miguelzamora210405@gmail.com}
  \\ \addr Grupo C212 \AND
\name Eric Reyes Mili\'an \email \href{mailto:ericreyesmilian05@gmail.com}{ericreyesmilian05@gmail.com}
\\ \addr Grupo C212 \AND
}

\tutors{\\
Doc. Copilot , \emph{by Github} \\
}

\ShortHeadings{Ecuaciones Diferenciales Ordinarias y Matemática Numérica}{Autores}

\begin{document}

\twocolumn[

\maketitle

\selectlanguage{spanish} 

\begin{abstract}

Este trabajo analiza tres problemas de ecuaciones diferenciales ordinarias. La Parte A estudia un modelo cinemático con aceleración variable ($\frac{dv}{dt} = 0.12t^2 + 0.6t$), obteniendo soluciones analíticas y comparando métodos numéricos. La Parte B investiga un sistema con bifurcación tipo horquilla ($\frac{dv}{dt} = rv - v^3$), analizando cómo el parámetro $r$ modifica cualitativamente el comportamiento del sistema, con puntos de equilibrio estables e inestables. La Parte C examina un sistema masa-resorte-amortiguador ($\frac{dx}{dt} = v$, $\frac{dv}{dt} = -\alpha v - \beta x$), clasificando su comportamiento según el discriminante $\Delta = \alpha^2 - 4\beta$ en tres casos: sobre-amortiguado, críticamente amortiguado y sub-amortiguado. Se visualiza el plano de fase para cada caso, demostrando estabilidad asintótica global cuando $\alpha, \beta > 0$.
\end{abstract}

\vspace{0.5cm}

\begin{enabstract}

  This paper analyzes three ordinary differential equation problems. Part A studies a kinematic model with variable acceleration ($\frac{dv}{dt} = 0.12t^2 + 0.6t$), obtaining analytical solutions and comparing numerical methods. Part B investigates a system with pitchfork bifurcation ($\frac{dv}{dt} = rv - v^3$), analyzing how parameter $r$ qualitatively changes the system's behavior, with stable and unstable equilibrium points. Part C examines a mass-spring-damper system ($\frac{dx}{dt} = v$, $\frac{dv}{dt} = -\alpha v - \beta x$), classifying its behavior according to the discriminant $\Delta = \alpha^2 - 4\beta$ into three cases: over-damped, critically damped, and under-damped. The phase plane is visualized for each case, demonstrating global asymptotic stability when $\alpha, \beta > 0$.
\end{enabstract}

\begin{keywords}
	Ecuaciones diferenciales ordinarias,
	Análisis cinemático,
	Estabilidad,
	Plano de fase,
	Sistemas dinámicos.
\end{keywords}

\begin{topics}
	Ecuaciones diferenciales ordinarias, Velocidad y aceleración.
\end{topics}

\vspace{0.8cm}
]

\section*{La Misión: Un Viaje a Contrarreloj}

Era una tarde nublada típica de MATCOM cuando \textbf{Alfonso}, \textbf{Eric} y \textbf{Miguel} recibieron un mensaje inesperado que cambiaría su destino académico (y, más importante aún, su acceso a las empanadas de la cafetería). 

Los profesores \textbf{Fernando} y \textbf{Amanda} de Matemática Numérica, junto con \textbf{Ania} y \textbf{Jackson} de Ecuaciones Diferenciales, se reunieron con ellos en el aula de laboratorio a las 11:00 am con una misión urgente:

\begin{center}
\textit{``Equipo: su misión de hoy es resolver la dinámica completa de un automóvil diesel, desde su aceleración hasta su comportamiento oscilatorio. Ah, y háganlo antes de que cierre la cafetería a las 12:00 pm. Si lo logran, las empanadas corren por cuenta de los profesores.''}
\end{center}

Los tres se miraron. Una hora de margen. La presión era real: si no terminaban, se quedaban sin empanadas. La estrategia fue dividir y conquistar.

\begin{description}
    \item[\textbf{Alfonso - Parte A: Cinemática Directa}] ``El arranque del auto''
    
    La misión de Alfonso era la más directa: resolver la dinámica de un auto diesel acelerado por una función $\frac{dv}{dt} = 0.12t^2 + 0.6t$. Tenía que integrar, calcular, visualizar. Un trabajo analítico donde la matemática fluía como el diesel en los cilindros del motor.
    
    Alfonso asintió con confianza. Siempre le habían gustado los números.
    
    \item[\textbf{Eric - Parte B: Bifurcación Pitchfork}] ``Cuando el motor decide''
    
    Eric recibió la ecuación $\frac{dv}{dt} = rv - v^3$, un sistema donde pequeños cambios en $r$ provocaban transformaciones radicales. Los profesores le advirtieron solemnemente:
    
    \begin{center}
    \textit{``Este sistema modela cuándo el motor se apaga, arranca o gira establemente. Y Eric... \textbf{no confíes en tu intuición para integrar cosas}.''} 
    \end{center}
    
    Eric sonrió nerviosamente. Era su punto débil más evidente. (Miguel murmuró: ``Al menos es honesto sobre sus limitaciones.'')
    
    \item[\textbf{Miguel - Parte C: Resorte Amortiguado}] ``El comportamiento oscilatorio''
    
    A Miguel le tocó el sistema más complejo: $\dot{x} = v$, $\dot{v} = -\alpha v - \beta x$. Un verdadero oscilador amortiguado que se comportaría de tres formas distintas según sus parámetros.
    
    Ania le explicó: ``Miguel, tienes que dominar el análisis de estabilidad, el plano de fase, los valores propios, el discriminante... todo el arsenal.''\
    
    Miguel tragó saliva. Pero asintió. Nunca le había faltado ambición.
\end{description}

\section{Parte A: Cinemática y Métodos Numéricos}
\label{sec:parte_a}

\subsection*{La carrera contrarreloj: Alfonso}

\textbf{11:05 am:} Alfonso ya estaba integrando. Primer integral: $\int (0.12t^2 + 0.6t) \, dt = 0.04t^3 + 0.3t^2$. Eso lo sabía de memoria.

En esta sección se presenta el análisis de un modelo cinemático con aceleración variable, descrito por una ecuación diferencial ordinaria de primer orden. Se estudia tanto la solución analítica como la implementación de diferentes métodos numéricos para su resolución.

\subsection{Planteamiento del Problema}

Se considera un móvil que parte del reposo desde el origen de coordenadas, cuya aceleración varía con el tiempo según la expresión:

\begin{equation}
a(t) = 0.12t^2 + 0.6t
\label{eq:aceleracion}
\end{equation}

El problema consiste en resolver la ecuación diferencial para la velocidad:

\begin{equation}
\frac{dv}{dt} = 0.12t^2 + 0.6t
\label{eq:edo_velocidad}
\end{equation}

Con las condiciones iniciales:
\begin{equation}
v(0) = 0 \text{ m/s}, \quad x(0) = 0 \text{ m}
\end{equation}

\subsection{Solución Analítica}

Integrando la ecuación diferencial para la velocidad, se obtiene:

\begin{equation}
v(t) = \int (0.12t^2 + 0.6t) \, dt = 0.04t^3 + 0.3t^2 + C
\end{equation}

Aplicando la condición inicial $v(0) = 0$, se determina que $C = 0$, por lo que:

\begin{equation}
v(t) = 0.04t^3 + 0.3t^2 \text{ (m/s)}
\label{eq:velocidad}
\end{equation}

Para obtener la posición, se integra la velocidad:

\begin{equation}
x(t) = \int v(t) \, dt = \int (0.04t^3 + 0.3t^2) \, dt = 0.01t^4 + 0.1t^3 + C'
\end{equation}

Con la condición inicial $x(0) = 0$, se obtiene $C' = 0$, resultando:

\begin{equation}
x(t) = 0.01t^4 + 0.1t^3 \text{ (m)}
\label{eq:posicion}
\end{equation}



\subsection{Valores Numéricos Clave}

La siguiente tabla muestra los valores de aceleración, velocidad y posición en momentos específicos:

\begin{table}[H]
\begin{center}
\begin{tabular}{|c|c|c|c|} \hline
\textbf{Tiempo (s)} & \textbf{Acel. (m/s²)} & \textbf{Vel. (m/s)} & \textbf{Pos. (m)} \\ \hline
0 & 0.00 & 0.00 & 0.00 \\ \hline
5 & 6.00 & 12.50 & 18.75 \\ \hline
10 & 18.00 & 70.00 & 200.00 \\ \hline
\end{tabular}
\caption{Valores cinemáticos en instantes clave. \label{tab:valores_clave}}
\end{center}
\end{table}


\subsection{Implementación de Métodos Numéricos}

Para resolver numéricamente la ecuación diferencial, se implementaron tres métodos:

\begin{enumerate}
    \item \textbf{Método de Euler:} El más simple pero menos preciso. Se basa en la aproximación lineal de la solución utilizando la pendiente en cada punto. La fórmula de recurrencia es:
    \begin{equation}
    v_{n+1} = v_n + h \cdot f(t_n, v_n)
    \end{equation}
    donde $h$ es el paso de integración y $f(t, v) = 0.12t^2 + 0.6t$ es la función que define la aceleración.
    
    \item \textbf{Método de Runge-Kutta de orden 2 (RK2):} Ofrece un equilibrio entre simplicidad y precisión. Utiliza una combinación ponderada de dos evaluaciones de la función para mejorar la aproximación:
    \begin{align}
    k_1 &= f(t_n, v_n) \\
    k_2 &= f(t_n + h, v_n + h \cdot k_1) \\
    v_{n+1} &= v_n + \frac{h}{2}(k_1 + k_2)
    \end{align}
    
    \item \textbf{Método de Runge-Kutta de orden 4 (RK4):} El más preciso de los tres. Utiliza cuatro evaluaciones de la función para obtener una aproximación de mayor orden:
    \begin{align}
    k_1 &= f(t_n, v_n) \\
    k_2 &= f(t_n + \frac{h}{2}, v_n + \frac{h}{2} \cdot k_1) \\
    k_3 &= f(t_n + \frac{h}{2}, v_n + \frac{h}{2} \cdot k_2) \\
    k_4 &= f(t_n + h, v_n + h \cdot k_3) \\
    v_{n+1} &= v_n + \frac{h}{6}(k_1 + 2k_2 + 2k_3 + k_4)
    \end{align}
\end{enumerate}

La implementación de estos métodos se realizó con diferentes pasos de integración para estudiar su convergencia y estabilidad. Se utilizó un paso base de $h = 0.1$ segundos para la comparación principal.

Los tres métodos presentan características distintivas: el \textbf{método de Euler} es de orden 1, requiere una evaluación por paso y tiene baja estabilidad y precisión. El \textbf{método RK2} es de orden 2, requiere dos evaluaciones por paso y ofrece estabilidad y precisión media. El \textbf{método RK4} es de orden 4, requiere cuatro evaluaciones por paso y proporciona alta estabilidad y precisión.

\subsection{Comparación de Métodos Numéricos}

La siguiente figura muestra una comparación entre los diferentes métodos numéricos implementados (Euler, RK2 y RK4) y la solución analítica:

\begin{figure}[H]
\centering
\includegraphics[width=0.9\columnwidth]{pictures/MethodsComparation.png}
\caption{Comparación entre métodos numéricos y la solución analítica}
\label{fig:comparacion_metodos}
\end{figure}

Para cuantificar la precisión de cada método, se calculó el error absoluto respecto a la solución analítica en diferentes puntos del intervalo $[0, 10]$ segundos. La tabla \ref{tab:errores_metodos} muestra estos errores para $t = 5$ s y $t = 10$ s con un paso de integración $h = 0.1$ s.

\begin{table}[H]
\begin{center}
\begin{tabular}{|c|c|c|c|} \hline
\textbf{Tiempo (s)} & \textbf{Euler} & \textbf{RK2} & \textbf{RK4} \\ \hline
5 & 0.625 & 0.031 & $<0.001$ \\ \hline
10 & 3.500 & 0.175 & 0.011 \\ \hline
\end{tabular}
\caption{Error absoluto de los métodos numéricos. \label{tab:errores_metodos}}
\end{center}
\end{table}

Se observa que el error del método de Euler crece significativamente con el tiempo, mientras que el método RK4 mantiene una precisión muy alta incluso al final del intervalo. El método RK2 presenta un comportamiento intermedio.

También se estudió la convergencia de los métodos al reducir el paso de integración. La tabla \ref{tab:convergencia} muestra cómo disminuye el error máximo al reducir el paso.

\begin{table}[H]
\begin{center}
\begin{tabular}{|c|c|c|c|} \hline
\textbf{Paso $h$} & \textbf{Euler} & \textbf{RK2} & \textbf{RK4} \\ \hline
0.1 & 3.500 & 0.175 & 0.011 \\ \hline
0.05 & 1.750 & 0.044 & $<0.001$ \\ \hline
0.01 & 0.350 & 0.002 & $<0.0001$ \\ \hline
\end{tabular}
\caption{Convergencia de los métodos numéricos. \label{tab:convergencia}}
\end{center}
\end{table}

\subsection{Análisis de Condicionamiento}

El número de condición del problema se visualiza para entender la sensibilidad de la solución a pequeñas perturbaciones:

\begin{figure}[H]
\centering
\includegraphics[width=0.9\columnwidth]{pictures/CondNumber.png}
\caption{Número de condición para el problema cinemático}
\label{fig:cond_number}
\end{figure}

El número de condición $\kappa(t)$ para este problema se define como:

\begin{equation}
\kappa(t) = \left| \frac{t \cdot f'(t)}{f(t)} \right|
\end{equation}

donde $f(t) = 0.12t^2 + 0.6t$ es la función de aceleración \eqref{eq:aceleracion}.

La tabla \ref{tab:cond_number} muestra los valores del número de condición en diferentes instantes de tiempo:

\begin{table}[H]
\begin{center}
\begin{tabular}{|c|c|} \hline
\textbf{Tiempo (s)} & \textbf{Núm. cond.} \\ \hline
1 & 1.29 \\ \hline
5 & 1.67 \\ \hline
10 & 1.82 \\ \hline
\end{tabular}
\caption{Número de condición del problema. \label{tab:cond_number}}
\end{center}
\end{table}

El análisis del número de condición muestra que el problema está bien condicionado en todo el intervalo de estudio, lo que significa que pequeñas perturbaciones en los datos de entrada producen pequeñas variaciones en la solución. Esto explica por qué los métodos numéricos de orden superior (RK2 y RK4) logran una excelente aproximación a la solución analítica.

\subsection{Visualización del Campo de Isoclinas}

El campo de isoclinas en el plano $(t, v)$ muestra cómo la solución sigue las pendientes del campo. Esta visualización permite comprender geométricamente cómo evoluciona la velocidad con el tiempo bajo la influencia de la aceleración variable.

\begin{figure}[H]
\centering
\includegraphics[width=0.9\columnwidth]{pictures/IsoCamp.png}
\caption{Campo de isoclinas para la ecuación diferencial de velocidad}
\label{fig:isoclinas}
\end{figure}

\subsection{Análisis del Comportamiento}

El análisis del movimiento revela características importantes:

\begin{itemize}
    \item \textbf{Aceleración variable:} Crece cuadráticamente con el tiempo, alcanzando $a = 18.0$ m/s² en $t = 10$ s.
    
    \item \textbf{Velocidad creciente:} Sigue una combinación cúbica-cuadrática, llegando a $v = 70.0$ m/s al final del intervalo estudiado. La velocidad promedio en los 10 segundos es $\bar{v} = 20.0$ m/s.
    
    \item \textbf{Desplazamiento:} La posición final es $x = 200$ m, siguiendo una trayectoria polinómica de grado 4.
\end{itemize}

\subsection*{11:35 am}

Alfonso había completado la Parte A. 200 metros en 10 segundos. La velocidad final de 70 m/s (252 km/h). Una aceleración de 18 m/s² al final.

``¡Ese carro anda más duro que el wifi del laboratorio!'', bromeó mientras presentaba sus gráficas de isoclinas.

Los profesores asintieron aprobatoriamente.

\section{Parte B: Bifurcación Tipo Horquilla}
\label{sec:parte_b}

\subsection*{La carrera contrarreloj: Eric}

\textbf{11:12 am:} Eric pedía ayuda para comprender qué significa un equilibrio inestable en $v=0$ cuando aparecen dos nuevos equilibrios en $v = \pm\sqrt{r}$. Fernando le mostraba el diagrama de bifurcación en la pizarra. 

``Es como cuando el sistema cambia de opinión,'' le explicaba Fernando. ``Antes solo había una opción (reposo). Luego hay dos opciones (velocidad positiva o negativa). El cambio ocurre en $r = 0$.''\

Eric asentía lentamente. Sí, eso tenía sentido.

En esta sección se analiza un sistema dinámico que presenta bifurcación tipo horquilla, modelado por la ecuación diferencial autónoma:

\begin{equation}
\frac{dv}{dt} = rv - v^3
\label{eq:bifurcacion}
\end{equation}

donde $r$ es un parámetro de bifurcación y $v$ es la variable de estado.

\subsection{Planteamiento del Problema}

La ecuación \eqref{eq:bifurcacion} describe sistemas que presentan transiciones cualitativas en su comportamiento dinámico cuando varía el parámetro $r$.

\subsection{Determinación de Puntos de Equilibrio}

Los puntos de equilibrio se encuentran cuando:

\begin{equation}
\frac{dv}{dt} = 0:
\end{equation}

\begin{equation}
rv - v^3 = 0 \Rightarrow v(r - v^2) = 0
\end{equation}

Esto produce:
\begin{enumerate}
    \item $v = 0$ (punto de equilibrio trivial para cualquier $r$)
    \item $v = \pm\sqrt{r}$ (existen solo cuando $r > 0$)
\end{enumerate}

\subsection{Análisis de Estabilidad}

La derivada de $f(v) = rv - v^3$ es:
\begin{equation}
f'(v) = r - 3v^2
\end{equation}

Análisis para $v = 0$:
\begin{equation}
f'(0) = r
\end{equation}

El análisis de estabilidad del punto $v = 0$ depende del signo del parámetro $r$: cuando $r < 0$, la derivada $f'(0) = r < 0$ indica que el punto es \textbf{estable}; cuando $r = 0$, la derivada es cero y el análisis lineal es \textbf{no concluyente}; cuando $r > 0$, la derivada $f'(0) = r > 0$ indica que el punto es \textbf{inestable}.

\paragraph{Análisis para $v = \pm\sqrt{r}$ (cuando $r > 0$):}
\begin{equation}
f'(\pm\sqrt{r}) = r - 3r = -2r < 0 \quad \text{(Siempre estables)}
\end{equation}

En resumen, la estabilidad de los puntos de equilibrio se comporta de la siguiente manera: el punto $v = 0$ es estable cuando $r < 0$ (con $f'(0) = r < 0$) e inestable cuando $r > 0$ (con $f'(0) = r > 0$). Los puntos $v = \pm\sqrt{r}$, que solo existen para $r > 0$, son siempre estables ya que $f'(\pm\sqrt{r}) = -2r < 0$.

\subsection{Diagrama de Bifurcación}

El diagrama de bifurcación visualiza cómo evolucionan los puntos de equilibrio y su estabilidad en función de $r$. Este diagrama es fundamental para entender cómo el parámetro $r$ controla la transición entre regímenes cualitativamente diferentes:

\begin{figure}[H]
\centering
\includegraphics[width=0.9\columnwidth]{pictures/BifurcationDiagram.png}
\caption{Diagrama de bifurcación tipo horquilla para $\frac{dv}{dt} = rv - v^3$. Se muestra claramente la transición en $r=0$ donde aparecen dos ramas simétricas de equilibrios estables.}
\label{fig:bifurcacion}
\end{figure}

\textbf{Descripción cualitativa:}

\begin{itemize}
    \item \textbf{Para $r < 0$:} Un único punto de equilibrio estable en $v = 0$. Las soluciones con cualquier condición inicial $v_0 \neq 0$ convergen a cero.
    
    \item \textbf{En $r = 0$:} El punto $v = 0$ cambia de estable a inestable. Punto de bifurcación crítico.
    
    \item \textbf{Para $r > 0$:} Aparecen dos nuevos puntos de equilibrio estables en $v = \pm\sqrt{r}$, mientras que el punto $v = 0$ se vuelve inestable. Las soluciones con condiciones iniciales pequeñas divergen desde el origen hacia los nuevos equilibrios.
\end{itemize}

\subsection{Retrato de Fases}

Para visualizar el comportamiento del sistema, se presentan análisis de estabilidad para diferentes valores de $r$:

\subsubsection{Caso 1: r = -1 (Pre-bifurcación)}

Para $r = -1$, el sistema presenta un único punto de equilibrio en $v = 0$, el cual es un nodo estable. Todas las trayectorias convergen hacia el origen.

\subsubsection{Caso 2: r = 0.5 (Post-bifurcación)}

Para $r = 0.5$, el sistema presenta tres puntos de equilibrio: $v = 0$ se convierte en un nodo inestable, mientras que aparecen dos nuevos equilibrios estables en $v = \pm\sqrt{0.5} \approx \pm 0.707$.

\subsection{Análisis No Lineal en r = 0}

En el punto crítico $r = 0$, la linealización no proporciona información. Se realiza un análisis no lineal:

\begin{equation}
\frac{dv}{dt} = -v^3
\end{equation}

Esta ecuación es separable:

\begin{equation}
\int v^{-3} dv = -\int dt \Rightarrow -\frac{1}{2v^2} = -t + C
\end{equation}

\begin{equation}
v(t) = \pm \frac{1}{\sqrt{2t + C}}
\end{equation}

Aunque hay equilibrio en $v = 0$, es marginalmente estable (no exponencial): las soluciones convergen lentamente como $v(t) \sim t^{-1/2}$.

\subsection{Interpretación Física}

La bifurcación tipo horquilla modela fenómenos tales como:

\begin{itemize}
    \item \textbf{Transiciones de fase:} Cambios cualitativos en estados físicos al variar un parámetro de control.
    
    \item \textbf{Pérdida de estabilidad:} El equilibrio inicial pierde estabilidad y se crean nuevos equilibrios simétricos.
    
    \item \textbf{Ruptura de simetría:} El sistema rompe su simetría respecto a $v = 0$ cuando $r > 0$.
\end{itemize}

\subsection{Comparación: Campo de Direcciones para Diferentes r}

El campo de direcciones muestra cómo cambian las trayectorias del sistema \eqref{eq:bifurcacion} para diferentes valores del parámetro de bifurcación $r$:

\begin{itemize}
    \item \textbf{Para $r = -1$}: El campo de direcciones converge hacia $v = 0$, indicando la estabilidad de este punto.
    \item \textbf{Para $r = 0$}: El punto crítico de bifurcación donde el comportamiento cambia cualitativamente.
    \item \textbf{Para $r = 0.5$}: El campo muestra divergencia desde $v = 0$ hacia $v = \pm\sqrt{0.5}$, indicando la aparición de dos nuevos equilibrios estables.
\end{itemize}

\subsection*{11:42 am}

Eric terminaba su análisis de bifurcación. Había comprendido que no necesitaba integrar: el análisis cualitativo del parámetro $r$ era suficiente. Una epifanía. (Amanda le guiñó un ojo: ``Ves, el análisis es más poderoso que la integración.'')

\section{Parte C: Análisis de Estabilidad y Plano de Fase}
\label{sec:parte_c}

\subsection*{La carrera contrarreloj: Miguel}

\textbf{11:20 am:} Miguel calculaba la matriz Jacobiana. Los valores propios salían complejos conjugados en el caso sub-amortiguado. Escribía:

\[
\lambda_{1,2} = -\frac{\alpha}{2} \pm i\frac{\sqrt{4\beta - \alpha^2}}{2}
\]

Eso significaba oscilaciones. Las trayectorias serían espirales. Lo visualizaba en su mente.

Consideramos el modelo matemático de un automóvil acoplado a un resorte con amortiguamiento, representado por el sistema de ecuaciones diferenciales:

\begin{equation}
\begin{cases}
\frac{dx}{dt} = v \\
\frac{dv}{dt} = -\alpha v - \beta x
\end{cases}
\label{eq:sistema_amortiguado}
\end{equation}

Donde $x$ es la posición del sistema, $v$ la velocidad y $\alpha,\beta>0$ son parámetros físicos: $\alpha$ es el coeficiente de amortiguamiento, responsable de la disipación de energía (reduce la amplitud y controla la tasa de decaimiento), y $\beta$ es la constante del resorte, que proporciona la fuerza restauradora y determina la rigidez y la frecuencia natural.

\subsection{Análisis de Puntos de Equilibrio}

\subsubsection{Determinación de Puntos Críticos}

Los puntos de equilibrio se determinan cuando el sistema permanece estático:

$$\frac{dx}{dt} = 0 \quad \text{y} \quad \frac{dv}{dt} = 0$$

\textbf{Resolución paso a paso:}

1. De $\frac{dx}{dt} = v = 0$ obtenemos: $v = 0$

2. De $\frac{dv}{dt} = -\alpha v - \beta x = 0$ y sustituyendo $v = 0$:
   $$- \alpha(0) - \beta x = 0 \Rightarrow x = 0$$

\textbf{Resultado:} (x, v) = (0, 0) es el único punto de equilibrio

\subsection{Análisis de Estabilidad Lineal}

Para el sistema vectorial
\[
\mathbf{F}(x,v)=\begin{pmatrix} v \\ -\alpha v - \beta x \end{pmatrix},
\]
las derivadas parciales son:
\[
\frac{\partial F_1}{\partial x}=0,\quad \frac{\partial F_1}{\partial v}=1,\qquad
\frac{\partial F_2}{\partial x}=-\beta,\quad \frac{\partial F_2}{\partial v}=-\alpha
\]

la matriz Jacobiana es:

\[
\mathbf{J}(x,v)=
\begin{pmatrix}
0 & 1 \\
-\beta & -\alpha
\end{pmatrix}.
\]

\subsection{Cálculo de Valores Propios}

\textbf{Ecuación característica:} $\det(\mathbf{J} - \lambda \mathbf{I}) = 0$

\begin{align}
\det\begin{pmatrix} -\lambda & 1 \\ -\beta & -\alpha - \lambda \end{pmatrix} &= (-\lambda)(-\alpha - \lambda) + \beta \\
&= \lambda^2 + \alpha\lambda + \beta = 0
\end{align}

\textbf{Aplicando la fórmula cuadrática:}
$$\lambda_{1,2} = \frac{-\alpha \pm \sqrt{\alpha^2 - 4\beta}}{2}$$

\textbf{Discriminante clave:} 
\begin{equation}
\Delta = \alpha^2 - 4\beta
\label{eq:discriminante}
\end{equation}

\subsection{Interpretación de los Valores Propios}

La interpretación de los valores propios se basa en tres factores fundamentales:

\begin{itemize}
  \item \textbf{Discriminante:} \eqref{eq:discriminante} controla la naturaleza de la convergencia y determina el tipo de amortiguamiento del sistema, pudiendo tomar cualquier valor real.
  
  \item \textbf{Parte real:} $\text{Re}(\lambda) = -\alpha/2$ controla la tasa de decaimiento y representa la velocidad de estabilización del sistema. Dado que $\alpha > 0$, esta parte real es siempre negativa, garantizando la estabilidad.
  
  \item \textbf{Parte imaginaria:} $\text{Im}(\lambda)$ controla la frecuencia de oscilación y representa la frecuencia natural del sistema. Solo está presente cuando $\Delta < 0$, es decir, en el caso sub-amortiguado.
\end{itemize}

\subsection{Clasificación de la Estabilidad}

La \textbf{estabilidad asintótica} se determina por el signo de las partes reales de los valores propios.

El comportamiento dinámico depende del \textbf{discriminante} $\Delta$ \eqref{eq:discriminante}. A continuación se presentan los tres casos posibles:

\subsubsection{Caso 1: $\Delta > 0$ (Sistema sobre-amortiguado)}

El amortiguamiento domina sobre la elasticidad del resorte ($\alpha^2 > 4\beta$). Los valores propios son dos reales distintos:
$$\lambda_1 = \frac{-\alpha + \sqrt{\alpha^2 - 4\beta}}{2}, \quad \lambda_2 = \frac{-\alpha - \sqrt{\alpha^2 - 4\beta}}{2}$$

Dado que $\beta > 0$: $\sqrt{\alpha^2 - 4\beta} < \alpha$, por tanto $\lambda_1 < 0$ y $\lambda_2 < 0$. El sistema presenta un \textbf{nodo estable}: sin oscilaciones, con regreso eficiente al equilibrio.

\subsubsection{Caso 2: $\Delta = 0$ (Sistema críticamente amortiguado)}

Representa el amortiguamiento óptimo ($\alpha^2 = 4\beta$), con valor propio real repetido:
$$\lambda = -\frac{\alpha}{2} < 0$$

El sistema presenta un \textbf{nodo crítico}: regreso más rápido sin oscilaciones, siendo el amortiguamiento óptimo para muchas aplicaciones prácticas.

\subsubsection{Caso 3: $\Delta < 0$ (Sistema sub-amortiguado)}

La elasticidad del resorte domina sobre el amortiguamiento ($\alpha^2 < 4\beta$). Los valores propios son un par complejo conjugado:
$$\lambda_{1,2} = -\frac{\alpha}{2} \pm i\frac{\sqrt{4\beta - \alpha^2}}{2}$$

Con parte real $\text{Re}(\lambda) = -\frac{\alpha}{2} < 0$ (decaimiento exponencial) e imaginaria $\text{Im}(\lambda) = \pm\frac{\sqrt{4\beta - \alpha^2}}{2}$ (frecuencia de oscilación). El sistema presenta un \textbf{foco estable}: oscilaciones amortiguadas alrededor del equilibrio.

\subsection{Verificación de Estabilidad Asintótica}

En todos los casos analizados, el sistema presenta estabilidad asintótica. Para el \textbf{caso sobre-amortiguado} ($\Delta > 0$), ambos valores propios son reales negativos ($\lambda_1, \lambda_2 < 0$), garantizando estabilidad asintótica. En el \textbf{caso críticamente amortiguado} ($\Delta = 0$), el valor propio repetido $\lambda = -\frac{\alpha}{2}$ es negativo, lo que también asegura estabilidad asintótica. Finalmente, en el \textbf{caso sub-amortiguado} ($\Delta < 0$), aunque los valores propios son complejos conjugados, su parte real $\text{Re}(\lambda) = -\frac{\alpha}{2}$ es negativa, manteniendo la estabilidad asintótica del sistema.

Dado que $\alpha > 0$, en los tres casos se cumple que $\text{Re}(\lambda_i) < 0$. Por el criterio de Lyapunov, el origen $(0,0)$ es \textbf{globalmente asintóticamente estable} para todo $\alpha, \beta > 0$.

\subsection{Plano de Fase}

El \textbf{plano de fase} $(x, v)$ representa todos los estados posibles del sistema. En este espacio bidimensional, el eje horizontal corresponde a la posición $x$ y el eje vertical a la velocidad $v$.

En cada punto $(x,v)$, el \textbf{vector de flujo} es:
\begin{equation}
\mathbf{F}(x,v) = \begin{pmatrix} v \\ -\alpha v - \beta x \end{pmatrix}
\end{equation}

\textbf{Pendiente de trayectorias:}
\begin{equation}
\frac{dv}{dx} = \frac{-\alpha v - \beta x}{v} \quad (v \neq 0)
\end{equation}

\subsubsection{Isoclinas Críticas}

El sistema presenta dos isoclinas críticas: la \textbf{isoclina horizontal}, donde $\frac{dv}{dt} = 0$, está definida por la ecuación $\alpha v + \beta x = 0$; la \textbf{isoclina vertical}, donde $\frac{dx}{dt} = 0$, está definida por $v = 0$.

\textbf{Significado físico de las isoclinas críticas:}

\begin{itemize}
  \item \textbf{Isoclina horizontal} ($v = -\frac{\beta}{\alpha}x$): La velocidad no cambia en esta línea ($dv/dt = 0$). Físicamente, las fuerzas están equilibradas instantáneamente. Las trayectorias cruzan esta línea horizontalmente.
  
  \item \textbf{Isoclina vertical} ($v = 0$): La posición no cambia en esta línea ($dx/dt = 0$). Físicamente, el sistema está momentáneamente en reposo. Las trayectorias cruzan esta línea verticalmente.
  
  \item \textbf{Intersección:} Donde se cruzan ambas isoclinas está el punto de equilibrio $(0,0)$, donde el sistema permanece estático indefinidamente.
\end{itemize}

\subsection{Visualización del Plano de Fase}

El plano de fase permite visualizar cómo el sistema \eqref{eq:sistema_amortiguado} se comporta bajo diferentes condiciones. Las trayectorias convergen hacia el punto de equilibrio $(0,0)$ desde cualquier condición inicial, demostrando la estabilidad global del sistema.

\subsubsection{Caso 1: Sistema Sobre-amortiguado ($\Delta > 0$)}

En este caso, el amortiguamiento es muy fuerte, dominando sobre la elasticidad del resorte. El sistema regresa al equilibrio sin oscilar, siguiendo trayectorias que se asemejan a las de un nodo estable.

\begin{figure}[H]
\centering
\includegraphics[width=0.9\columnwidth]{pictures/DampedOscillator_Overdamped.png}
\caption{Plano de fase para el caso sobre-amortiguado. Las trayectorias convergen directamente al equilibrio sin oscilaciones.}
\label{fig:fase_sobre_amortiguado}
\end{figure}

\textbf{Características observadas:}
\begin{itemize}
  \item Las trayectorias no oscilan alrededor del equilibrio
  \item Convergencia directa sin sobreimpulso
  \item Comportamiento característico de un nodo estable
  \item Ocurre cuando $\alpha^2 > 4\beta$
\end{itemize}

\subsubsection{Caso 2: Sistema Críticamente Amortiguado ($\Delta = 0$)}

Este es el caso límite entre el comportamiento sobre-amortiguado y sub-amortiguado. Representa el amortiguamiento óptimo que permite el retorno más rápido al equilibrio sin oscilaciones.

\begin{figure}[H]
\centering
\includegraphics[width=0.9\columnwidth]{pictures/DampedOscillator_Critically.png}
\caption{Plano de fase para el caso críticamente amortiguado. Convergencia óptima sin oscilaciones.}
\label{fig:fase_critico}
\end{figure}

\textbf{Características observadas:}
\begin{itemize}
  \item Convergencia más rápida que el caso sobre-amortiguado
  \item Sin oscilaciones alrededor del equilibrio
  \item Valor propio repetido $\lambda = -\alpha/2$
  \item Comportamiento de nodo crítico
  \item Condición: $\alpha^2 = 4\beta$
\end{itemize}

\subsubsection{Caso 3: Sistema Sub-amortiguado ($\Delta < 0$)}

En este caso, la elasticidad del resorte domina sobre el amortiguamiento. El sistema oscila alrededor del equilibrio con amplitud decreciente, presentando el comportamiento característico de un foco estable.

\begin{figure}[H]
\centering
\includegraphics[width=0.9\columnwidth]{pictures/DampedOscillator_Underdamped.png}
\caption{Plano de fase para el caso sub-amortiguado. Las trayectorias espiralan hacia el equilibrio.}
\label{fig:fase_sub_amortiguado}
\end{figure}

\textbf{Características observadas:}
\begin{itemize}
  \item Oscilaciones amortiguadas alrededor del equilibrio
  \item Trayectorias en forma de espiral convergente
  \item Valores propios complejos: $\lambda = -\frac{\alpha}{2} \pm i\frac{\sqrt{4\beta - \alpha^2}}{2}$
  \item Tasa de decaimiento exponencial: $e^{-\alpha t/2}$
  \item Ocurre cuando $\alpha^2 < 4\beta$
\end{itemize}

\subsection{Comparación de los Tres Casos}

Los tres regímenes de amortiguamiento presentan características distintivas que permiten su clasificación. El \textbf{sistema sobre-amortiguado} ($\Delta > 0$) tiene valores propios reales distintos, comportándose como un nodo estable sin oscilaciones y con retorno lento al equilibrio. El \textbf{sistema críticamente amortiguado} ($\Delta = 0$) presenta un valor propio real repetido, formando un nodo crítico que representa el amortiguamiento óptimo: el retorno más rápido posible sin oscilaciones. Finalmente, el \textbf{sistema sub-amortiguado} ($\Delta < 0$) exhibe valores propios complejos conjugados, manifestándose como un foco estable con oscilaciones amortiguadas y velocidad de retorno intermedia.

\subsection{Conclusión del Análisis de Estabilidad}

El análisis completo del sistema \eqref{eq:sistema_amortiguado} demuestra que para $\alpha, \beta > 0$, el sistema es \textbf{globalmente asintóticamente estable}. Independientemente de las condiciones iniciales, el sistema siempre converge al punto de equilibrio $(0, 0)$, y la energía se disipa completamente. La visualización en el plano de fase confirma que todos los casos de amortiguamiento resultan en trayectorias convergentes al equilibrio.

\subsection*{11:50 am}

Miguel presentaba sus tres casos de amortiguamiento en el plano de fase. Sobre-amortiguado: un nodo sin oscilaciones. Críticamente amortiguado: el punto óptimo. Sub-amortiguado: una bella espiral convergente.

Jackson miraba los diagramas y sonreía: ``Esto es teoría de sistemas dinámicos en su máxima expresión.''\

\textbf{11:57 am:} Los tres equipos habían terminado. Todo el análisis estaba completo. Tres minutos antes del cierre.

\section*{El epílogo}

Fernando escribió un mensaje en la pizarra:

\begin{center}
\textit{``Excelente trabajo, equipo. Las empanadas están ganadas. Pero lo más importante es que han aprendido que las ecuaciones diferenciales no son solo símbolos en un papel: describen el movimiento, la estabilidad, la transformación de sistemas reales. Hoy resolvieron la dinámica de un auto diesel. Mañana podrían resolver el comportamiento de un reactor nuclear o el crecimiento de una población. Que disfruten su premio.''}
\end{center}

Ania añadió con una sonrisa:

\begin{center}
\textit{``Y recuerden: integración, análisis cualitativo y linealización. Esas tres herramientas les abrirán puertas a cualquier sistema dinámico que encuentren.''}
\end{center}

Alfonso, Eric y Miguel salieron corriendo hacia la cafetería. Llegaron exactamente a las 11:59 am. Las empanadas estaban calientes.

\label{end}

\end{document}